\vspace*{2em}
\begin{center}
	\subsection{Symposium 1 : Conception et évaluation de tableaux de bord d’apprentissage}
\end{center}


\vspace{1em}
\begin{center}
	\begin{tabular}{@{}c@{}}
		Sébastien Iksal\textsuperscript{1}, Madjid Sadallah\textsuperscript{2}, Katia Quelennec\textsuperscript{3}\\
		\textsuperscript{1}LIUM, Le Mans Université\\
		\textsuperscript{2}Lab-STICC, IMT Atlantique\\
		\textsuperscript{3}Université de Lille
	\end{tabular}
\end{center}

\vspace{2em}

Le domaine des Learning Analytics offre de nouvelles perspectives d’analyse des différents processus d’apprentissage. Dans cet atelier, nous poursuivons les travaux sur le développement d’outils permettant l’appropriation des Learning Analytics par les utilisateurs potentiels et la capture des besoins utilisateurs.

L’atelier Quels tableaux de bord pour les acteurs de l’éducation ? organisé dans le cadre de la conférence EIAH2017 a permis d’identifier les différentes dimensions et les enjeux liés aux tableaux de bord d’apprentissage. Depuis, un outil de conception participative a été proposé et utilisé dans des séances de travail par différentes équipes de la communauté, démontrant l’intérêt pour ce type de démarches.

Dans le cadre de la conférence EIAH 2021, un deuxième atelier Conception participative de tableaux de bord d’apprentissage a été réalisé avec l’objectif d’identifier les usages potentiels de tels outils et de dégager des propositions pour développer de nouveaux outils à disposition de la communauté. Suite à cela une nouvelle version de l’outil de conception participative PaDLAD V2 a été proposé et utilisé. Dans le cadre de cet atelier, nous souhaitons avancer sur la capitalisation de ce qui a été mené dans les différentes expériences de conception et de réfléchir à l’évaluation, l’adaptation et le processus qualité lié aux tableaux de bord d’apprentissage. L’objectif final étant la création du groupe de travail ATIEF sur le sujet.

\newpage

\vspace*{2em}
\begin{center}
	\subsection{Symposium 2 : Cadres théoriques, état de l’art pour les EIAH ?}
\end{center}

\vspace{1em}
\begin{center}
	\begin{tabular}{@{}c@{}}
		Nadine Mandran\textsuperscript{1}, Nour El Mawas\textsuperscript{2}\\
		\textsuperscript{1}LIG, Université de Grenoble Alpes \\
		\textsuperscript{2}CIREL, Université de Lille
	\end{tabular}
\end{center}

\vspace{2em}

Les notions de cadres théoriques, d’état de l’art scientifique, de positionnement, de travaux connexes, de modèles d’analyses sont des outils pour construire d’une part une problématique et identifier des manques auxquels la recherche peut apporter des contributions. Ces différentes dimensions sont importantes dans toutes disciplines. Cependant dans le cadre d’un travail au cœur de plusieurs disciplines, le sens et la finalité de ces concepts ne sont pas toujours partagés. Cette méconnaissance entraîne parfois des difficultés pour co-élaborer des projets de recherche. De plus, la confusion entre ces termes met parfois en difficulté les doctorant.e.s.

L’objectif de l’atelier est d’échanger autour de ces concepts selon les disciplines présentes dans les EIAH,  de mieux se comprendre sur ces concepts, d’échanger sur les méthodes de mobilisation de la littérature et de construction de l’état de l’art.

Des activités seront organisées pour que les doctorant.e.s et les chercheurs seniors confrontent leurs points de vue et leurs méthodes.

\newpage

\vspace*{2em}
\begin{center}
	\subsection{Symposium 3 : La notion de compétence pour les EIAH}
\end{center}

\vspace{1em}
\begin{center}
	\begin{tabular}{@{}c@{}}
		Mathieu Vermeulen\textsuperscript{1}, Nathalie Guin\textsuperscript{2}\\
		\textsuperscript{1}IMT Nord-Europe\\
		\textsuperscript{2}LIRIS, Université de Lyon 1
	\end{tabular}
\end{center}

\vspace{2em}

La notion de compétence est devenue centrale dans nombre de situations liées à la formation et à l’apprentissage, que ce soit en formation initiale, continue ou professionnelle. L’intégration de l’approche par compétences dans les cycles primaires et secondaires est aujourd’hui effective et sa mise en place est en cours dans le supérieur. Outre le besoin d’information et de formation à ce nouveau paradigme, le mot compétence en tant que tel engendre des incompréhensions d’ordre sémantique ou fonctionnelles.  Le caractère protéiforme de cette notion rend son appropriation difficile, en particulier parce qu’elle est liée au contexte de son utilisation.

En ce qui concerne les EIAH, la multiplicité des définitions et le croisement des nombreuses expertises, ainsi que la diversité des cadres épistémologiques de la recherche en EIAH, impactent les travaux menés par les chercheurs du domaine, conduisant à un certain flou autour de la notion de compétence. Pour autant, de récents projets (ANR xCALE, ANR COMPER, iSite ULNE APACHES, etc.) travaillent sur son intégration dans les artefacts informatiques avec des objectifs variés : la modélisation des compétences, l’assistance à l’évaluation de celles-ci, la personnalisation des parcours des apprenants, l’accompagnement à la mise en place des approches par compétences, etc. De fait, le besoin d’offrir un cadre favorisant une meilleure appropriation du concept de compétence, et donc une intégration efficiente dans les travaux de recherche en EIAH, semble aujourd’hui devenir indispensable.

Cet atelier fait suite au Séminaire APACHES d’octobre 2021 et propose, au travers de présentations et d’activités participatives, de travailler sur cette notion de compétence, sur son intégration aux EIAH et en particulier sur la co-élaboration d’une grille de questions permettant aux chercheurs de positionner la compétence dans leurs travaux en fonction de leurs objectifs et de leurs contextes.

\newpage

\vspace*{2em}
\begin{center}
	\subsection{Symposium 4 : Adaptation et génération dans les EIAH}
\end{center}

\vspace{1em}
\begin{center}
	\begin{tabular}{@{}c@{}}
		Pierre Laforcade\textsuperscript{1}, Sébastien Jolivet\textsuperscript{2}, Marie Lefevre\textsuperscript{3}\\
		\textsuperscript{1}LIUM, Le Mans Université\\
		\textsuperscript{2}LDAR, Université Paris Diderot\\
		\textsuperscript{3}LIRIS, Université Lyon 1
	\end{tabular}
\end{center}

\vspace{2em}

L’adaptation dans un EIAH est une activité complexe, pluri-disciplinaire, qui nécessite d’être appréhendée en prenant en compte ses nombreuses dimensions (didactique, pédagogique, ludique, motivationnelle, informatique...), ses perspectives (cibles, sources et objectifs de l’adaptation), ainsi que les différents acteurs concernés.

L’atelier proposé dans le cadre de RJC’22 s’intéresse plus particulièrement aux dimensions ludiques et motivationnelles. L’atelier est organisé sous la forme d’un symposium autour de 4 invités :

\begin{itemize}
	\item Élise Lavoué (LIRIS), chercheure reconnue pour ses travaux sur les thématiques de l’adaptation et de la ludification en EIAH ; elle présentera ces derniers travaux autour de la gamification adaptative et l’engagement des apprenants.
	\item Bertrand Marne (ICAR), chercheur reconnu pour ses travaux en ingénierie des jeux sérieux ;
	\item Bérénice Lemoine (LIUM), doctorante abordant la génération d’activités d’apprentissage ludiques adaptées dans un jeu sérieux ;
	\item Luca Pelissero-Witoslawski (HEUDIASYC), doctorant abordant la génération dynamique de situations stressantes en environnement virtuel d’apprentissage.
\end{itemize}