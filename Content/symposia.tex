\vspace*{2em}
\begin{center}
	\subsection{Atelier 1 : Construire son état de l’art dans une littérature exubérante}
\end{center}


\vspace{1em}
\begin{center}
	\begin{tabular}{@{}c@{}}
		Nour El Mawas\textsuperscript{1}, Mariem Jaouadi\textsuperscript{2}, Nadine Mandran\textsuperscript{3}\\
		\textsuperscript{1}CREM, Université de Lorraine\\
		\textsuperscript{2}TECFA, Université de Genève\\
		\textsuperscript{3}LIG, CNRS
	\end{tabular}
\end{center}

\vspace{2em}

Pour conduire toute recherche, un des travaux est la constitution d’un corpus d’articles scientifiques pour connaître les travaux existants et ainsi identifier des manques auxquels le doctorant apportera une réponse via une contribution scientifiquement construite et évaluée. Or le doctorant est aujourd’hui confronté à une littérature scientifique exubérante.

En effet, le nombre d’articles dans la littérature scientifique a considérablement augmenté ces dernières années (en 2018 4,18 millions, en 2022 5,14 millions d’articles ). Ce phénomène s’explique par les politiques des institutions de recherche incitant les personnels à fortement publier, la pression des classements comme celui de Shangaï, etc.

Ce foisonnement est rendu possible grâce à la technologie numérique qui permet d'une part de conduire plus rapidement des recherches et d'autre part d’avoir acc ès à un plus grand nombre d'articles sans jouer les « rats de bibliothèques ». Si cette accessibilité est une réelle avancée elle peut mettre en difficulté les doctorants qui se demandent souvent comment tout lire ? Est-ce que j’aurais vraiment tout lu ?

Ce sont toujours les premières questions que les doctorants posent lors de formations en école doctorale et ce quel que soit la discipline. Les directeurs de thèse sont eux aussi parfois démunis pour répondre à ces questions. Les doctorants sont parfois aussi confrontés à une terminologie diverse et polysémique (e.g. cadres théoriques, état de l’art, fondements théoriques, etc.) qui les met en difficulté quand ils doivent eux même utiliser ces vocables pour la structuration de leur pensée et pour la rédaction.

Étant donné que la qualité de ce travail de sélection de la littérature, de l’analyse des articles et du positionnement théorique font partie intégrante du travail de recherche, il est important de proposer une méthode et des outils aux doctorants pour réaliser ce travail. Cette méthode doit aussi assurer la traçabilité de ce travail pour le rendre opposable aux demandes des relecteurs ou des rapporteurs pour une thèse.

\newpage

\vspace*{2em}
\begin{center}
	\subsection{Atelier 2 : Modélisations de l’activité de programmation (savoirs et savoir-faire, compétences, capacités…)}
\end{center}

\vspace{1em}
\begin{center}
	\begin{tabular}{@{}c@{}}
		Sebastien Jolivet\textsuperscript{1}, Yvan Peter\textsuperscript{2}\\
		\textsuperscript{1}TECFA \& IUFE, Université de Genève\\
		\textsuperscript{2}CRISTAL - Université de Lille
	\end{tabular}
\end{center}

\vspace{2em}

Comprendre ce qu’est l’activité de programmation est un questionnement ancien. Diverses approches ont guidé ce questionnement : quelle est l’activité cognitive ? Quelles sont les compétences de programmation ? Quels sont les savoirs et savoir-faire liés à l’apprentissage de la programmation ?

Diverses propositions pour modéliser et représenter ces éléments existent déjà : référentiels de compétences, référentiels de praxéologies, modèle COMPER, etc. Les motivations menant à la production de tels modélisations sont aussi très diverses : objectifs orientés métier / formation ; exploitations dans un EIAH (adaptive learning…) ; outils pour l’enseignement ; outils d’analyse – description de ressources ; etc. Ces objectifs différents amènent des différences : niveau de modélisation, grain des compétences…

Cet atelier a pour objectif de mettre en regard ses différents travaux, d’origines et de motivations diverses, pour essayer d’identifier les convergences possibles (formalisation des référentiels ; processus de construction ; articulations entre des référentiels de niveaux de granularité différents ; etc.). La matinée aura été l'occasion de présenter un certain nombre de travaux liés à la modélisation, sous différentes formes, de l'activité de programmation.

L'objectif de l'après-midi est d'aller au-delà de l'identification du champs de l'existant et de proposer un cadre pour mieux identifier les convergences, les articulations possibles, les mutualisations et dessiner le champs des possibles. Remarque : pour participer l'après midi il est nécessaire (en tous cas fortement souhaitable) de participer à l'atelier du matin.